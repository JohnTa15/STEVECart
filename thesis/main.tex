% !Tex root = E:/sxolh/SmartCart/SmartCart/thesis/main.tex
documentclass[a4paper,11pt]{article}

\usepackage[margin=25mm]{geometry}
\usepackage{setspace}
\setstretch{1.15}
\usepackage{fontspec}
\setmainfont{Times New Roman}

\title{Smart Cart Management System using Microservices and Container-based Virtualization}
\author{John Tarnaras}

\usepackage{titlesec}
\newfontfamily\arial{Arial}
\titleformat{\section}
  {\arial\bfseries\fontsize{12pt}{12pt}\selectfont}
    {\thesection}{1em}{}
    \titelformat{\subsection}
   {arial\bfseries\fontsize{12pt}{12pt}\selectfont}
    {\thesubsection}{1em}{}

\begin{document}
\maketitle
\end{document}
\begin{description}
    \item[Αντικείμενο της παρούσας διπλωματικής εργασίας είναι η μελέτη και έρευνα του έξυπνου καροτσιού (smart
cart), με σκοπό τη διευκόλυνση του καταναλωτή κατά τη διαδικασία αγορών εντός του καταστήματος.
Το καρότσι θα είναι εξοπλισμένο με αισθητήρες RFID, υπερηχητικούς αισθητήρες απόστασης (ultrasound
distance sensors), ζυγαριές και άλλους αισθητήρες. Θα μελετηθούν αρχικά οι απαιτήσεις του συστήματος
αυτού, τόσο ως προς τις προσωπικές προτιμήσεις του καταναλωτή και την ασφάλεια του καλαθιού του, όσο
και ως προς την προστασία του καταστήματος. Στη συνέχεια θα υλοποιηθεί πρόγραμμα βασισμένο σε
μικροϋπηρεσίες, Docker, Kubernetes ή αντίστοιχα, με σκοπό την αποτελεσματική υποστήριξη προς τον
καταναλωτή και το κατάστημα. Τέλος, θα πραγματοποιηθούν οι ανάλογες προσομοιώσεις σε πραγματικό
χρόνο των δεδομένων των αισθητήρων, ώστε να επιβεβαιωθεί η ακρίβεια και αποτελεσματικότητα της
εργασίας.] 
\end{description}

\end{document}